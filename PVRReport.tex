%!TEX program = xelatex
\documentclass[UTF8,zihao=5]{ctexart} %ctex包的article


\usepackage[hidelinks]{hyperref}%超链接,自动加到目录里面



\title{{\bfseries\rmfamily\Huge{变分重构方法测试}}}
\author{周涵宇 2022310984}
\date{}

\usepackage[a4paper]{geometry}
\geometry{left=0.75in,right=0.75in,top=1in,bottom=1in}%纸张大小和页边距

\usepackage[
UseMSWordMultipleLineSpacing,
MSWordLineSpacingMultiple=1.5
]{zhlineskip}%office风格的行间距

\usepackage{fontspec}
\setmainfont{Times New Roman}
\setsansfont{Source Sans Pro}
\setmonofont{Latin Modern Mono}
\setCJKmainfont{SimSun}[AutoFakeBold=true]
% \setCJKmainfont{仿宋}[AutoFakeBold=true]
\setCJKsansfont{黑体}[AutoFakeBold=true]
\setCJKmonofont{DengXian}[AutoFakeBold=true]

\setCJKfamilyfont{kaiti}{楷体}
\newfontfamily\CM{Cambria Math}


% \usepackage{indentfirst} %不工作 怎样调整ctex的段首缩进大小呢

\usepackage{fancyhdr}
\pagestyle{fancy}
\lhead{}
\chead{}
\rhead{}
\lfoot{}
\cfoot{\thepage}
\rfoot{}
\renewcommand{\headrulewidth}{1pt} %改为0pt即可去掉页眉下面的横线
\renewcommand{\footrulewidth}{1pt} %改为0pt即可去掉页脚上面的横线
\setcounter{page}{1}


% \usepackage{bm}

\usepackage{amsmath,amsfonts}
\usepackage{array}
\usepackage{enumitem}
\usepackage{unicode-math}

% \usepackage{titlesec} % it subverts the ctex titles
\usepackage{titletoc}


% titles in toc:
\titlecontents{section}
              [2cm]
              {\sffamily\zihao{5}\mdseries}%
              {\contentslabel{3em}}%
              {}%
              {\titlerule*[0.5pc]{-}\contentspage\hspace*{1cm}}

\titlecontents{subsection}
              [3cm]
              {\rmfamily\mdseries\zihao{5}}%
              {\contentslabel{3em}}%
              {}%
              {\titlerule*[0.5pc]{-}\contentspage\hspace*{1cm}}

\titlecontents{subsubsection}
              [4cm]
              {\rmfamily\mdseries\zihao{5}}%
              {\contentslabel{3em}}%
              {}%
              {\titlerule*[0.5pc]{-}\contentspage\hspace*{1cm}}
\renewcommand*\contentsname{\hfill \sffamily\mdseries 目录 \hfill}

\ctexset{
    section={   
        % name={前面,后面},
        number={\arabic{section}.},
        format=\sffamily\raggedright\zihao{4}\mdseries,
        indent= {0em},
        aftername = \hspace{0.5em},
        beforeskip=1ex,
        afterskip=1ex
    },
    subsection={   
        % name={另一个前面,另一个后面},
        number={\arabic{section}.\arabic{subsection}.}, %如果只用一个数字而非1.1
        format=\rmfamily\raggedright\mdseries\zihao{5},%正体字体,不加粗,main字体,五号字
        indent = {2em}, %缩进
        aftername = \hspace{0.5em},
        beforeskip=1ex,
        afterskip=1ex
    },
    subsubsection={   
        % name={另一个前面,另一个后面},
        number={\arabic{section}.\arabic{subsection}.\arabic{subsubsection}.}, %默认的 1.1.1
        format=\rmfamily\raggedright\mdseries\zihao{5},%无衬线字体,加粗,sans字体,五号字
        indent = {2em}, %缩进
        aftername = \hspace{0.5em},  %名字和标题间插入字符(此处是空白)
        beforeskip=1ex, %空行
        afterskip=1ex
    }
}

\usepackage{float}
\usepackage{graphicx}
\usepackage{multirow}
\usepackage{multicol}
\usepackage{caption}
\usepackage{subcaption}
\usepackage{cite}
\usepackage{physics}


%part、section、subsection、subsubsection、paragraph、subparagraph
\newcommand{\bm}[1]{{\mathbf{#1}}}
\newcommand{\trans}[0]{^\mathrm{T}}
\newcommand{\tran}[1]{#1^\mathrm{T}}
\newcommand{\hermi}[0]{^\mathrm{H}}
\newcommand{\conj}[1]{\overline{#1}}
\newcommand*{\av}[1]{\left\langle{#1}\right\rangle}
\newcommand*{\avld}[1]{\frac{\overline{D}#1}{Dt}}
\newcommand*{\pd}[2]{\frac{\partial #1}{\partial #2}}
\newcommand*{\pdcd}[3]{\frac{\partial^2 #1}{\partial #2 \partial #3}}
\newcommand*{\inc}[0]{{\Delta}}

\newcommand*{\xx}[0]{\bm{x}}
\newcommand*{\uu}[0]{\bm{u}}
\newcommand*{\vv}[0]{\bm{v}}
\newcommand*{\g}[0]{\bm{g}}
\newcommand*{\nb}[0]{{\nabla}}

\newcommand*{\mean}[1]{\overline{#1}}



\begin{document}

\maketitle

\section{基函数}

\subsection{Local泰勒基函数}

物理空间的位置:
$$
    \xx = [x,y,z]\trans
$$

二维为例,泰勒基函数为
$$
    [\hat{\varphi_i}]=
    \begin{bmatrix}
        1    \\
        X    \\
        Y    \\
        X^2  \\
        XY   \\
        Y^2  \\
        X^3  \\
        X^2Y \\
        XY^2 \\
        Y^3
    \end{bmatrix}_i
$$
其中
$$
    \bm{X}=\begin{bmatrix}
        X \\Y\\Z
    \end{bmatrix}=\begin{bmatrix}
        \inc X &  & \\ & \inc Y & \\ & & \inc Z
    \end{bmatrix}^{-1}\begin{bmatrix}
        x-x_c \\y-y_c\\z-z_c
    \end{bmatrix}
$$
经过零均值化后是此前VR的基函数用法。

\subsection{正交基函数}

设有某组基函数$\varphi_i$,以及一个函数空间上的内积$\innerproduct{\cdot}{\cdot}$,
那么可找到一个系数矩阵$\mathcal{O}$,使得
$$
    \psi_i = \sum_j{\mathcal{O}_{ij}\varphi_j}
$$
为单位正交基函数。事实上,由于
$$
    \delta_{ij}=\innerproduct{\psi_i}{\psi_j}
    =\innerproduct{\mathcal{O}_{il}\varphi_l}{\mathcal{O}_{jm}\varphi_m}
    =\mathcal{O}_{il}\mathcal{O}_{jm}\innerproduct{\varphi_l}{\varphi_m}
$$
(上式中$l,m$求和),不妨要求$\mathcal{O}_{ij}$可逆,则有:
$$
    \mathcal{O^{-1}}_{li}\mathcal{O^{-1}}_{mi}=\innerproduct{\varphi_l}{\varphi_m}
$$
也就是说
$$
    [\mathcal{O}]^{-1}[\mathcal{O}]^{-\mathrm{T}}=[\innerproduct{\varphi_l}{\varphi_m}]
$$
需要对对称阵$[\innerproduct{\varphi_l}{\varphi_m}]$进行某个对称分解。
最简单的方案是Gram-Schmidt过程,等价于对度量矩阵进行Cholesky分解,获得的下三角矩阵的逆即为
$[\mathcal{O}]$。问题是需要保证原始基函数的独立性,否则Gram-Schmidt过程不存在。

\section{投影后构建的变分重构}

定义($i$单元)cell-wise的内积:
$$
    \innerproduct{f_1}{f_2}_{\Omega_i}=\frac{1}{\Omega_i}\int_{\Omega_i}{f_1f_2d\Omega}
$$
对于每个单元上某组零均值正交基函数$\psi_i^l$(此处下标为单元,上标为基维度),定义重构:
$$
    u_i=\mean{u}_i + \sum_{l>0}{u_i^l\psi_i^l}
$$
这样,$\psi_i^0=1$恰为正交基的一部分,非常数部分与之正交且
自身为单位大小。
则可以定义基于正交投影(系数)的泛函:
$$
    I_f=
    % \left(\mean{u_i-u_j}\right)_i^2\Omega_i^2 + \left(\mean{u_i-u_j}\right)_j^2\Omega_j^2
    % +
    \sum_{m}
    {
        w_{b,f,m}
        \innerproduct{u_i-u_j}{\psi_i^m}_{\Omega_i}^2
    }
    +
    \sum_{m}
    {
        w_{b,f,m}
        \innerproduct{u_i-u_j}{\psi_j^m}_{\Omega_j}^2
    }
$$
简单推导:
$$
    \begin{aligned}
        I_f & =
        \sum_{m}
        {
            w_{b,f,m}
            \innerproduct{\mean{u}_i + \sum_{l>0}{u_i^l\psi_i^l} - \mean{u}_j - \sum_{l>0}{u_j^l\psi_j^l} }{\psi_i^m}_{\Omega_i}^2
        }
        +
        \sum_{m}
        {
            w_{b,f,m}
            \innerproduct{\mean{u}_i + \sum_{l>0}{u_i^l\psi_i^l} - \mean{u}_j - \sum_{l>0}{u_j^l\psi_j^l} }{\psi_j^m}_{\Omega_j}^2
        }          \\
            & =
        \sum_{m}
        {w_{b,f,m}\left(
            \innerproduct{\mean{u}_i - \mean{u}_j}{\psi_i^m}_{\Omega_i}
            +
            \innerproduct{\sum_{l>0}{u_i^l\psi_i^l} - \sum_{l>0}{u_j^l\psi_j^l} }{\psi_i^m}_{\Omega_i}
            \right)^2}
        \\
            & +
        \sum_{m}
        {w_{b,f,m}\left(
            \innerproduct{\mean{u}_i - \mean{u}_j}{\psi_j^m}_{\Omega_j}
            +
            \innerproduct{\sum_{l>0}{u_i^l\psi_i^l} - \sum_{l>0}{u_j^l\psi_j^l} }{\psi_j^m}_{\Omega_j}
        \right)^2} \\
            & =
        \sum_{m}
        {w_{b,f,m}\left(
            (\mean{u}_i - \mean{u}_j)\delta_{m0}
            +
            u_i^m(1-\delta_{m0})
            - \sum_{l>0}u_j^l\innerproduct{\psi_j^l}{\psi_i^m}_{\Omega_i}
            \right)^2}
        \\
            & +
        \sum_{m}
        {w_{b,f,m}\left(
            (\mean{u}_j - \mean{u}_i)\delta_{m0}
            +
            u_j^m(1-\delta_{m0})
            - \sum_{l>0}u_i^l\innerproduct{\psi_i^l}{\psi_j^m}_{\Omega_j}
        \right)^2} \\
            & =
        \left[
            (\mean{u}_i - \mean{u}_j)
            - \sum_{l>0}u_j^l\innerproduct{\psi_j^l}{1}_{\Omega_i}
            \right]^2
        +
        \left[
            (\mean{u}_j - \mean{u}_i)
            - \sum_{l>0}u_i^l\innerproduct{\psi_i^l}{1}_{\Omega_j}
        \right]^2  \\
            & +
        \sum_{m>0}
        {w_{b,f,m}\left(
            u_i^m
            - \sum_{l>0}u_j^l\innerproduct{\psi_j^l}{\psi_i^m}_{\Omega_i}
            \right)^2}
        +
        \sum_{m>0}
        {w_{b,f,m}\left(
            u_j^m
            - \sum_{l>0}u_i^l\innerproduct{\psi_i^l}{\psi_j^m}_{\Omega_j}
        \right)^2} \\
    \end{aligned}
$$
那么
$$
    \begin{aligned}
        \frac{1}{2}\pd{I_f}{u_i^k}
         & =
        \left[
            (\mean{u}_j - \mean{u}_i)
            - \sum_{l>0}u_i^l\innerproduct{\psi_i^l}{1}_{\Omega_j}
            \right]\left[
            - \innerproduct{\psi_i^k}{1}_{\Omega_j}
            \right]
        \\
         & +
         w_{b,f,k} u_i^k
        - w_{b,f,k}\sum_{l>0}u_j^l\innerproduct{\psi_j^l}{\psi_i^k}_{\Omega_i}
        +
        \sum_{m>0}
        w_{b,f,m}\left[
            {\left(
                    u_j^m
                    - \sum_{l>0}u_i^l\innerproduct{\psi_i^l}{\psi_j^m}_{\Omega_j}
                    \right)\left(
                    - \innerproduct{\psi_i^k}{\psi_j^m}_{\Omega_j}
                    \right)}
        \right]                                                                                                 \\
         & =
         w_{b,f,k}u_i^k +
        \sum_{m>0}\sum_{l>0}w_{b,f,m}
        \innerproduct{\psi_i^k}{\psi_j^m}_{\Omega_j}
        \innerproduct{\psi_i^l}{\psi_j^m}_{\Omega_j}u_i^l +
        \sum_{l>0}\innerproduct{\psi_i^k}{1}_{\Omega_j}\innerproduct{\psi_i^l}{1}_{\Omega_j}u_i^l \\
         &
        - \sum_{m>0}w_{b,f,m}\innerproduct{\psi_i^k}{\psi_j^m}_{\Omega_j}u_j^m
        - \sum_{l>0}w_{b,f,k}\innerproduct{\psi_j^l}{\psi_i^k}_{\Omega_i}u_j^l
        + (\mean{u}_i - \mean{u}_j)\innerproduct{\psi_i^k}{1}_{\Omega_j}
        , k>0
    \end{aligned}
$$
组装向量$k>0$,$\uu_i=[u_i^k]$,则上式为
$$
\frac{1}{2}\pd{I_f}{\uu_i} = A_{f,i} \uu_i - B_{ij} \uu_j - 
b_{ij}(\mean{u}_j - \mean{u}_i ) 
$$
其中系数矩阵与向量为:
$$
[A_{f,i}]_{kl} = w_{b,f,k} \delta_{kl} +
\sum_{m>0}  
w_{b,f,m}
\innerproduct{\psi_i^k}{\psi_j^m}_{\Omega_j}
\innerproduct{\psi_i^l}{\psi_j^m}_{\Omega_j}
+
\innerproduct{\psi_i^k}{1}_{\Omega_j}
\innerproduct{\psi_i^l}{1}_{\Omega_j}
$$
$$
[B_{ij}]_{kl} = 
w_{b,f,l}\innerproduct{\psi_i^k}{\psi_j^l}_{\Omega_j}
+
w_{b,f,k}\innerproduct{\psi_j^l}{\psi_i^k}_{\Omega_i}
$$
$$
[b_{ij}]_{k} = 
\innerproduct{\psi_i^k}{1}_{\Omega_j}
$$
记
$$
[\mathcal{B}_{i,j}]_{km}=\innerproduct{\psi_i^k}{\psi_j^m}_{\Omega_j}
$$
则矩阵表示下:
$$
[A_{f,i}] = I W_{b,f} +
[\mathcal{B}_{i,j}] W_{b,f} [\mathcal{B}_{i,j}]\trans
+
[b_{ij}][b_{ij}]\trans
$$
$$
[B_{ij}] = 
[\mathcal{B}_{i,j}]W_{b,f}
+
W_{b,f}[\mathcal{B}_{i,j}]\trans
$$
其中$W_{b,f}=\text{diag}(w_{b,f,1},w_{b,f,2},...w_{b,f,N_{base}-1})$
为(试验)基权重。

因此,如果已经获知每个单元的正交基函数,同时可进行临单元延拓并求内积,
即可计算$[\mathcal{B}_{i,j}]$以及$[b_{ij}]$,进一步获得重构系数。







\bibliography{refs}{}
\bibliographystyle{unsrt}


% \section*{附录}

% 本文使用的计算代码都在
% \href{https://github.com/harryzhou2000/HW_ACFD}{Github的Git Repo(点击前往)}。




















% \section{SECTION 节}

% 一个

% \subsection{SUBSECTION 小节}

% 示例

% \subsubsection{SUBSUBSECTION 小节节}

% 字体字号临时调整:
% {
%    \sffamily\bfseries\zihao{3} 哈哈哈哈哈 abcde %三号 sans系列字体(一开始设置的) 加粗
%    %只对大括号范围内的后面的字有用,在标题、题注里面同样
% }
% { 
%    \CJKfamily{kaiti}\zihao{5}\itshape 哈哈哈哈哈 abcde%三号 kaiti(一开始设置的, 斜体(英文有变)
%    %只对大括号范围内的后面的字有用,在标题、题注里面同样
% }

% 一大堆一大堆一大堆一大堆一大堆一大堆一大堆一大堆一大堆一大堆
% 一大堆一大堆一大堆一大堆一大堆一大堆一大堆一大堆一大堆一大堆一大堆一大堆
% 一大堆一大堆一大堆一大堆一大堆一大堆一大堆一大堆一大堆一大堆一大堆一大堆
% 一大堆一大堆一大堆一大堆一大堆一大堆一大堆一大堆一大堆一大堆一大堆一大堆

% \begin{center}
%     居中的什么乱七八糟东西
% \end{center}


% 一个列表:
% \begin{itemize}
%     \item asef
%     \item[\%] asdf
%     \item[\#] aaa
% \end{itemize}

% 一个有序列表:
% \begin{enumerate}
%     \item asef
%     \item[\%\%] asdf
%     \item aaa
% \end{enumerate}

% 一个嵌套列表,考虑缩进:
% \begin{enumerate}[itemindent=2em] %缩进
%     \item asef \par asaf 东西东西东西东西东西东西东西东西东西东西东西东西东西东西东西东西东西东西东西东西东西东西东西东西,
%           F不是不是不是不是不是不是不是不是不是不是不是不是不是不是不是
%           \begin{itemize}[itemindent=2em]  %缩进
%               \item lalala
%               \item mamama
%           \end{itemize}
%     \item asdf
%     \item aaa
% \end{enumerate}

% \section{SECTION}

% 图片排版:

% \begin{figure}[H]
%     \begin{minipage}[c]{0.45\linewidth}  %需调整
%         \centering
%         \includegraphics[width=8cm]{RAM_O2_4660.png}  %需调整
%         \caption{第一个图}
%         \label{fig:a}
%     \end{minipage}
%     \hfill %弹性长度
%     \begin{minipage}[c]{0.45\linewidth}  %需调整
%         \centering
%         \includegraphics[width=8cm]{RAM_O4_4660.png}  %需调整
%         \caption{第二个图}
%         \label{fig:b}
%     \end{minipage}
% \end{figure}

% figure的选项为“htbp”时,会自动浮动,是“H”则和文字顺序严格一些。

% \begin{figure}[H]
%     \begin{minipage}[c]{0.45\linewidth}  %需调整
%         \centering
%         \includegraphics[width=8cm]{RAM_O2_4660.png}  %需调整
%         \label{fig:x}
%     \end{minipage}
%     \hfill %弹性长度
%     \begin{minipage}[c]{0.45\linewidth}  %需调整
%         \centering
%         \includegraphics[width=8cm]{RAM_O4_4660.png}  %需调整
%         \label{fig:y}
%     \end{minipage}
%     \caption{第三个图}
% \end{figure}

% \begin{figure}[H]
%     \centering
%     \includegraphics[width=8cm]{RAM_O4_4660.png}  %需调整
%     \label{fig:c}
%     \caption{第四个图}
% \end{figure}



% \subsection{SUBSECTION}

% 关于怎么搞表格:

% \begin{table*}[htbp]
%     \footnotesize
%     \begin{center}
%         \caption{一端力矩载荷下的结果\fontsize{0pt}{2em}} %需要学习统一设置;0代表不变?
%         \label{表2}
%         \begin{tabular}{|c|c|c|c|c|c|c|}
%             \hline
%             节点数                              & 积分方案              & 单元数                & $h=1m$                & $h=0.1m$              & $h=0.05m$             & $h=0.01m$             \\
%             \hline
%             \multirow{6}{*}{2}                  & \multirow{3}{*}{精确} & 1                     & 4.235294117647059E-08 & 1.406250000000000E-06 & 2.862823061630218E-06 & 1.439654482924097E-05 \\
%             \cline{3-7}
%                                                 &                       & 10                    & 5.975103734439814E-08 & 4.235294117646719E-05 & 1.800000000000410E-04 & 1.406249999999849E-03 \\
%             \cline{3-7}
%                                                 &                       &
%             10000                               & 5.999999915514277E-08 & 5.999996622448291E-05 & 4.799989509752562E-04 & 5.999793702477535E-02                                                 \\
%             \cline{2-7}
%                                                 & \multirow{3}{*}{减缩} & 1                     & 6.000000000000001E-08 & 5.999999999999972E-05 & 4.799999999999911E-04 & 6.000000000003492E-02 \\
%             \cline{3-7}
%                                                 &                       & 10                    & 6.000000000000071E-08 & 5.999999999999142E-05 & 4.799999999995399E-04 & 5.999999999903294E-02 \\
%             \cline{3-7}
%                                                 &                       & 10000                 & 6.000000112649221E-08 & 5.999999234537814E-05 & 4.799997501925065E-04 & 6.000037607984510E-02 \\
%             \hline

%             \multirow{6}{*}{3}                  & \multirow{3}{*}{精确} & 1                     & 6.000000000000003E-08 & 6.000000000000202E-05 & 4.800000000000831E-04 & 6.000000000056749E-02 \\
%             \cline{3-7}
%                                                 &                       & 10                    & 5.999999999999932E-08 & 6.000000000004190E-05 & 4.800000000000206E-04 & 6.000000001613761E-02 \\
%             \cline{3-7}
%                                                 &                       & 10000                 & 6.000000013769874E-08 & 5.999989495410481E-05 & 4.799942099727246E-04 & 6.000263852944890E-02 \\
%             \cline{2-7}
%                                                 & \multirow{3}{*}{减缩} & 1                     & 6.000000000000002E-08 & 6.000000000000267E-05 & 4.800000000000754E-04 & 5.999999999989982E-02 \\
%             \cline{3-7}
%                                                 &                       & 10                    & 5.999999999999899E-08 & 5.999999999987338E-05 & 4.799999999947916E-04 & 5.999999998625345E-02 \\
%             \cline{3-7}
%                                                 &                       & 10000                 & 5.999999728157785E-08 & 5.999994914321980E-05 & 4.800008377474699E-04 & 5.999472246346305E-02 \\
%             \hline

%             \multicolumn{3}{|c|}{欧拉-伯努利解} & 6.000000000000000E-08 & 6.000000000000000E-05 & 4.800000000000000E-04 & 6.000000000000000E-02                                                 \\
%             \hline
%         \end{tabular}
%     \end{center}
% \end{table*}

% 多行、多列表格的示例,基本思想是,多列的那个东西放在多列的最上面一格,下面的行要用\&来空开,也就是\&的数目
% 和普通表格一样,是列数减一;
% 多列的部分的话,就是每行内的操作,相应的\&就少了,见最后一行。

% tabular的“|c|c|c|c|c|c|c|”,意思是,竖线-居中-竖线-居中-竖线……,可以选择省略一些竖线;
% 每行之间的hline,代表贯通的横线,cline是有范围的横线。

% \subsubsection{SUBSUBSECTION}

% newcommand可以用来定义新指令,似乎基本上就是字符串替换……不太懂,总之在公式里面可以用,
% 外面也经常用。






% 公式这么写:
% \begin{equation}
%     \begin{aligned}
%         \frac{aa(x^1+x^2)}{\sqrt{x^1x^2}}
%         \nabla\times\uu
%         = & u_{j;m}\g^m\times\g^j
%         =u_{j;m}\epsilon^{mjk}\g_k
%         =u_{j,m}\epsilon^{mjk}\g_k                           \\
%         = & \frac{1}{\sqrt{g}}\left|
%         \begin{matrix}
%             \g_1       & \g_2       & \g_3       \\
%             \partial_1 & \partial_2 & \partial_3 \\
%             u_1        & u_2        & u_3
%         \end{matrix}
%         \right|
%         =\frac{\sqrt{x^1x^2}}{aa(x^1+x^2)}
%         \left|
%         \begin{matrix}
%             \g_1                        & \g_2                        & \g_3       \\
%             \partial_1                  & \partial_2                  & \partial_3 \\
%             u^1\frac{a^2(x^1+x^2)}{x^1} & u^2\frac{a^2(x^1+x^2)}{x^2} & u^3
%         \end{matrix}
%         \right|                                              \\
%         = & \frac{\sqrt{x^1x^2}}{aa(x^1+x^2)}
%         [[\g_1\,\g_2\,\g_3]]
%         diag\left(
%         u^3_{,2}-u^2_{,3}\frac{a^2(x^1+x^2)}{x^2},\,
%         u^1_{,3}\frac{a^2(x^1+x^2)}{x^1}-u^3_{,1},\, \right. \\
%           & \left.
%         u^2_{,1}\frac{a^2(x^1+x^2)}{x^2}+u^2\frac{a^2}{x^2}
%         -
%         u^1_{,2}\frac{a^2(x^1+x^2)}{x^1}-u^1\frac{a^2}{x^1}
%         \right)                                              \\
%         = & \frac{\sqrt{x^1x^2}}{aa(x^1+x^2)}
%         [[\bm{e}_1\,\bm{e}_2\,\bm{e}_3]]
%         \left[\begin{array}{ccc} a & -a & 0\\ \frac{a\,x^{2}}{\sqrt{x^{1}\,x^{2}}} & \frac{a\,x^{1}}{\sqrt{x^{1}\,x^{2}}} & 0\\ 0 & 0 & 1 \end{array}\right]              \\
%           & diag\left(
%         u^3_{,2}-u^2_{,3}\frac{a^2(x^1+x^2)}{x^2},\,
%         u^1_{,3}\frac{a^2(x^1+x^2)}{x^1}-u^3_{,1},\, \right. \\
%           & \left.
%         u^2_{,1}\frac{a^2(x^1+x^2)}{x^2}+u^2\frac{a^2}{x^2}
%         -
%         u^1_{,2}\frac{a^2(x^1+x^2)}{x^1}-u^1\frac{a^2}{x^1}
%         \right)
%     \end{aligned}
%     \label{eq:curlu}
% \end{equation}

% 如果不想带编号的公式(或者图表),用 equation* 这种环境。

% 引用,如果是引用的图表,就用表\ref{表2},图\ref{fig:a}这种,代码里是用label定义的标签来引用,
% 编号是自动生成的。公式引用一般写成:\eqref{eq:curlu}。目前这些引用自动会有超链接,反正有那个包自动
% 好像就会有……呜呜呜也不知道是怎么做到的,先这么用吧。

% \paragraph{PARA}

% 引用文献用\\cite这些,要用bibtex,暂时不做。

% \subparagraph{SUBPARA}

\end{document}